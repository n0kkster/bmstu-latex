\usepackage{adjustbox}
\usepackage{amsfonts}
\usepackage{amsmath}
\usepackage{array}
\usepackage[english,russian]{babel}
\usepackage{blindtext}
\usepackage{bmstu-title}
\usepackage{caption}
\usepackage{chngcntr}
\usepackage{cmap}
\usepackage{csvsimple}
\usepackage{enumitem} 
\usepackage[14pt]{extsizes}
\usepackage{float}
\usepackage[T2A]{fontenc}
\usepackage{geometry}
\usepackage{graphicx}
\usepackage{hhline}
\usepackage[pdftex]{hyperref}
\usepackage{indentfirst}
\usepackage[utf8]{inputenc}
\usepackage{lastpage}
\usepackage{listings}
\usepackage{multirow}
\usepackage{pdfpages}
\usepackage{ragged2e}
\usepackage{rotating}
\usepackage{setspace}
\usepackage{siunitx} % Данного пакета нет в раннерах на АА. Нужен для выравнивания чисел в таблицах по точке.
\usepackage{tabularx}
\usepackage[figure,table]{totalcount}
\usepackage{threeparttable}
\usepackage{titlesec}[explicit]
\usepackage{ulem} % Данного пакета нет в раннерах на АА. Нужен для генерации титульника. Альтернатива --- сгенерить титул локально и пришить как ПДФ.
\usepackage{url}
\usepackage[table]{xcolor}
\usepackage{xparse}

% Настройка полей
% ============================
\geometry{left=30mm}
\geometry{right=15mm}
\geometry{top=20mm}
\geometry{bottom=20mm}
% ============================

% Форматирование секций
% ============================
\titleformat{name=\section,numberless}[block]{\normalfont\large\bfseries\centering}{}{0pt}{}
\titleformat{\section}[block]{\normalfont\large\bfseries}{\thesection}{1em}{}

\titleformat{\subsection}[hang]
{\bfseries\large}{\thesubsection}{1em}{}
\titlespacing\subsection{\parindent}{*2}{*2}

\titleformat{\subsubsection}[hang]
{\bfseries\large}{\thesubsubsection}{1em}{}
\titlespacing\subsubsection{\parindent}{*2}{*2}

\titlespacing\section{\parindent}{*4}{*4}
% ============================

% Общее форматирование
% ============================
% Полуторный интервал
\onehalfspacing

% Убираем увеличенные пробелы после точек
\frenchspacing

% Красная строка
\setlength\parindent{1.25cm}
% ============================

% Сокрытие цветных рамок вокруг ссылок
\hypersetup{hidelinks}

% Настройка списков
% ============================
\setenumerate[0]{label=\arabic*)}
\renewcommand{\labelitemi}{---}
% ============================

% Настройки листингов
% ============================
\lstset{
	basicstyle=\footnotesize\ttfamily\linespread{0.8},
	language=C,
	numbers=left,
	numbersep=5pt,
	xleftmargin=17pt,
	numbersep=5pt,
	frame=single,
	tabsize=4,	
	captionpos=b,
	breaklines=true,
	breakatwhitespace=true,	
	escapeinside={\#*}{*)},	
	inputencoding=utf8x,
	backgroundcolor=\color{white},
	numberstyle=\tiny,
	keywordstyle=\color{blue},
	stringstyle=\color{red!90!black}, 
	commentstyle=\color{green!50!black}
}
\lstset{
	morekeywords={size_t, likely}
}
\lstset{
	literate=
	{а}{{\selectfont\char224}}1
	{б}{{\selectfont\char225}}1
	{в}{{\selectfont\char226}}1
	{г}{{\selectfont\char227}}1
	{д}{{\selectfont\char228}}1
	{е}{{\selectfont\char229}}1
	{ё}{{\"e}}1
	{ж}{{\selectfont\char230}}1
	{з}{{\selectfont\char231}}1
	{и}{{\selectfont\char232}}1
	{й}{{\selectfont\char233}}1
	{к}{{\selectfont\char234}}1
	{л}{{\selectfont\char235}}1
	{м}{{\selectfont\char236}}1
	{н}{{\selectfont\char237}}1
	{о}{{\selectfont\char238}}1
	{п}{{\selectfont\char239}}1
	{р}{{\selectfont\char240}}1
	{с}{{\selectfont\char241}}1
	{т}{{\selectfont\char242}}1
	{у}{{\selectfont\char243}}1
	{ф}{{\selectfont\char244}}1
	{х}{{\selectfont\char245}}1
	{ц}{{\selectfont\char246}}1
	{ч}{{\selectfont\char247}}1
	{ш}{{\selectfont\char248}}1
	{щ}{{\selectfont\char249}}1
	{ъ}{{\selectfont\char250}}1
	{ы}{{\selectfont\char251}}1
	{ь}{{\selectfont\char252}}1
	{э}{{\selectfont\char253}}1
	{ю}{{\selectfont\char254}}1
	{я}{{\selectfont\char255}}1
	{А}{{\selectfont\char192}}1
	{Б}{{\selectfont\char193}}1
	{В}{{\selectfont\char194}}1
	{Г}{{\selectfont\char195}}1
	{Д}{{\selectfont\char196}}1
	{Е}{{\selectfont\char197}}1
	{Ё}{{\"E}}1
	{Ж}{{\selectfont\char198}}1
	{З}{{\selectfont\char199}}1
	{И}{{\selectfont\char200}}1
	{Й}{{\selectfont\char201}}1
	{К}{{\selectfont\char202}}1
	{Л}{{\selectfont\char203}}1
	{М}{{\selectfont\char204}}1
	{Н}{{\selectfont\char205}}1
	{О}{{\selectfont\char206}}1
	{П}{{\selectfont\char207}}1
	{Р}{{\selectfont\char208}}1
	{С}{{\selectfont\char209}}1
	{Т}{{\selectfont\char210}}1
	{У}{{\selectfont\char211}}1
	{Ф}{{\selectfont\char212}}1
	{Х}{{\selectfont\char213}}1
	{Ц}{{\selectfont\char214}}1
	{Ч}{{\selectfont\char215}}1
	{Ш}{{\selectfont\char216}}1
	{Щ}{{\selectfont\char217}}1
	{Ъ}{{\selectfont\char218}}1
	{Ы}{{\selectfont\char219}}1
	{Ь}{{\selectfont\char220}}1
	{Э}{{\selectfont\char221}}1
	{Ю}{{\selectfont\char222}}1
	{Я}{{\selectfont\char223}}1
}
% ============================

% Настройка подписей 
% ============================
\captionsetup[figure]{
	name={Рисунок},
	justification=centering,
	labelsep=endash,
	singlelinecheck=on
}
\captionsetup[table]{
	name={Таблица},
	justification=raggedright,
	labelsep=endash,
	singlelinecheck=off
}
% \captionsetup[lstlisting]{
% 	name={Листинг},
% 	justification=centering,
% 	labelsep=endash,
% 	singlelinecheck=on
% }
% ============================

% Метки в библиографии
% ============================
\makeatletter
\renewcommand\@biblabel[1]{#1.}
\makeatother
% ============================

% Переопределение \underset для лучшего отображения нижних индексов в тексте
\renewcommand{\underset}[2]{\ensuremath{\mathop{\mbox{#2}}\limits_{\mbox{\scriptsize #1}}}}

% Подчёркнутый текст с пояснением снизу
\NewDocumentCommand{\ulinetext}{O{3cm} O{c} m m}
{\underset{#3}{\uline{\makebox[#1][#2]{#4}}}}

% Секция без номера, но с записью в оглавлении
\newcommand{\numberlesssection}[1]{\phantomsection\section*{#1}\addcontentsline{toc}{section}{#1}}

% Вставка картинок
\newcommand{\img}[3] {
	\begin{figure}[h!]
		\center{\includegraphics[height=#1]{schemes/#2}}
		\caption{#3}
		\label{img:#2}
	\end{figure}
}

% Inline код
\newcommand{\code}[1]{{\small \ttfamily #1}}

% Центрирование в таблице, растянутой по ширине страницы
\newcolumntype{Y}{>{\centering\arraybackslash}X}
